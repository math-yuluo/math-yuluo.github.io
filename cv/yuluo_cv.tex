\documentclass[margin,line,pifont,palatino,courier]{res}

\usepackage{pifont}
\usepackage[latin1] { inputenc}

%\topmargin .5in
%\oddsidemargin -.5in
%\evensidemargin -.5in
%\textwidth=6.0in
 \textheight=9.0in
%\itemsep=0in
%\parsep=0in
\usepackage{fancyhdr}
%\topmargin=0in
%\textheight=8.5in
\pagestyle{fancy}
\renewcommand{\headrulewidth}{0pt}
\fancyhf{}
%\cfoot{\thepage}
%\lfoot{\textit{\footnotesize Research Statement}}



\newenvironment{list1}{
  \begin{list}{\ding{113}}{%
      \setlength{\itemsep}{0in}
      \setlength{\parsep}{0in} \setlength{\parskip}{0in}
      \setlength{\topsep}{0in} \setlength{\partopsep}{0in}
      \setlength{\leftmargin}{0.17in}}}{\end{list}}
\newenvironment{list2}{
  \begin{list}{$\bullet$}{%
      \setlength{\itemsep}{0in}
      \setlength{\parsep}{0in} \setlength{\parskip}{0in}
      \setlength{\topsep}{0in} \setlength{\partopsep}{0in}
      \setlength{\leftmargin}{0.2in}}}{\end{list}}

\begin{document}

\name{Yu LUO \vspace*{.1in}}

\begin{resume}

\section{\sc Contact Information}

\vspace{.05in}
\begin{tabular}{@{}p{2.75in}p{2in}}
Department of Mathematics & (608)949-2961 \\
University of Wisconsin - Madison                      & \verb+yluo237@wisc.edu+\\
518 Van Vleck Hall                 & 
math-yuluo.github.io\\
480 Lincoln Drive               & \\
Madison, WI 53706 USA               & \\
\end{tabular}


\section{\sc Research Interests}
Arithmetic Algebraic Geometry

\section{\sc Education}

{\bf University of Wisconsin - Madison}\\
\vspace*{-.1in}
\begin{list1}
\item[] Ph.D.~in Mathematics (In progress)


\item[] M.A.~in Mathematics, May 2021
\end{list1}

{\bf Zhejiang University}\\
\vspace*{-.1in}
\begin{list1}
\item[] B.S.~in Mathematics, May 2020

\end{list1}






%\section{\sc Publications}

%J. Doe,  \textit{A simple piston problem in one dimension,}
%submitted to Nonlinearity (May 1998).


%A. Smith and J. Doe,  \textit{Semiclassical generalization
%of the Darboux-Christoffel formula,} J. Math. Phys. \textbf{43}
%(1996), no. 10, 4668-4680.


%\section{\sc Invited Talks}


\section{\sc Honors and Awards}

\begin{tabular}{@{}p{0.8in}p{4in}}
2016--2017 & Government Scholarship\\
           & Zhejiang Government  \\
           
2017--2018 & Government Scholarship\\
           & Zhejiang Government  \\

2018--2019 & A scholarship equivalent to the Government Scholarship\\
\end{tabular}


\bigskip



\section{\sc Services}

\begin{tabular}{@{}p{0.4in}p{0.3in}p{4in}}
    2022 & Spring & Coorganizers, GAGS - Graduate Algebraic Geometry Seminar \\

    2022 & Spring & Mentor, Undergraduate Directed Reading Program - Baoyi WANG\\
    
    2021 & Fall & Mentor, Master Program in UW-Madison Math Department.\\
    
         &      & Ruoxi LI (Currently in UIUC Math PhD), Weiman YUAN (Currently Vanderbilt Univ. Bio. Endineering), Aditya Phukon. \\


	2020 & Spring & Organizer, Algebraic Geometry Learning Seminar\\
	
	
	2019 & Fall & Organizer, Commutative Algebra Learning Seminar\\
	
	2019 & Spring & Organizer, Undergraduate Algebraic Geometry Seminar\\
\end{tabular}



\section{\sc Scientific Research Experience}

\begin{tabular}{@{}p{0.8in}p{4in}}

2020--2020 & Vector Bundle of Algebraic Curves. \\
& \hspace{0.2in} Advisor: Fei Yu, Department of Mathematics, Zhejiang University.\\


2019--2019 & Weil Conjecture for Algebraic Curves \\
& \hspace{0.2in} Advisor: Xinyi Yuan, Beijing International Center for Mathematical Research, Peking University. \\


2019--2019 & A simple question about simple Coxeter elements \\
& \hspace{0.2in} Advisor: Xuhua He, Mathematics of the Institute of Mathematical Sciences,  Chinese University of Hong Kong.\\

2019--2019 & Weil Conjecture for  Elliptic Curves \\
& \hspace{0.2in} Advisor: Xinyi Yuan, Beijing International Center for Mathematical Research ,Peking University.\\

\end{tabular}


\section{\sc Seminar Talks}

\emph{Riemann-Hilbert Correspondence}, Graduate Algebraic Geometry Seminar,
University of Wisconsin Madison. (Feberary  2022)


\emph{Stacks for Kindergarteners}, Graduate Algebraic Geometry Seminar,
University of Wisconsin Madison. (December  2021)

\emph{Grothendieck Complexes and Flat Base Change}, Graduate Deformation Theory Seminar,
University of Wisconsin Madison. (October  2021)

\emph{Construction of Quot scheme and flat family}, Graduate Deformation Theory Seminar,
University of Wisconsin Madison. (October  2021)

\emph{Intersection Theory and Grothendieck Riemann Roch}, Graduate Algebraic Curve Reading Seminar,
University of Wisconsin Madison. (April  2021)

\emph{The Zariski Tangent space of $W^r_d(C)$ and $G^r_d(C)$}, Graduate Algebraic Curve Reading Seminar,
University of Wisconsin Madison. (March 2021)


\emph{First Properties of Schemes}, Undergraduate Algebraic Geometry Reading Seminar,
Zhejiang University. (May 2019)

\emph{Schemes}, Undergraduate Algebraic Geometry Reading Seminar,
Zhejiang University. (April 2019)

\emph{Nonsingular Varieties}, Undergraduate Algebraic Geometry Reading Seminar,
Zhejiang University. (March 2019)

\emph{Rational Maps}, Undergraduate Algebraic Geometry Reading Seminar,
Zhejiang University. (March 2019)

\emph{Affine Varieties and Dimension Theory}, Undergraduate Algebraic Geometry Reading Seminar,
Zhejiang University. (March 2019)

\emph{Covering Space}, Undergraduate Algebraic Topology Reading Seminar,
Zhejiang University. (December 2018)

\emph{Abstract theory of weights}, Undergraduate Lie Algebra Reading Seminar,
Zhejiang University. (November 2018)

\emph{Root system and classification of Dynkin diagram}, Undergraduate Lie Algebra Reading Seminar,
Zhejiang University. (November 2018)

\emph{Fundamental group and singular homology}, Undergraduate Algebraic Topology Reading Seminar,
Zhejiang University. (October 2018)




\bigskip




\section{\sc Relevant \\ Skills}

\begin{tabular}{@{}p{0.8in}p{6in}}

Languages:& English, Mandarin\\

\end{tabular}





\end{resume}
\end{document}
